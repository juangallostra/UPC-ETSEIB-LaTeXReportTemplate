% Example section
\chapter{Binary Knapsack problem with constraints}
\label{ch:Example}

Let us consider a set of 10 items from which we are to select some number of them to be carried in a knapsack. The weights and profits of each item are showed in the following table jointly with the knapsack capacity:
\begin{lstlisting}[language=Matlab,frame=none]
% Object data
w=[ 2  5  1  4  5  6  5  8  5  4  3  4  5  1  5  4  3  5  2  1]; 		% weights
v=[ 6  5  3  5  3  5  4  4  2  3  4  8  6  5  4  1  2  4  9  5]; 		% values
W = 30;    																% Knapsack capacity	
\end{lstlisting}
Furthermore we have some additional constraints:
\begin{enumerate}
\item We have to select at least one of the items 1, 2 or 3 and at least one of the items 4, 5 or 6.
\item We cannot select together items 14 and 15.
\item We have to select exactly two items from the set {1, 4, 16, 17, 18}.
\item The solution cannot contain the four items 1, 5, 10 and 20 simultaneously.
\item The solution must contain less than ten items.
\end{enumerate}
The objective is to choose the set of items that fits in the knapsack, satisfying all constraints simultaneously, and maximizing the total profit.

\section{Formulate the problem as a binary program (write the objective function and all the constraints as functions of $x_i$, $w_i$, $v_i$ and $W$).}

The first step to formulate a problem as a binary program is to identify if the problem is well suited for being expressed as a binary program. A program is normally suited for being expressed as a binary program when the optimal solution can be stated as a set of binary decisions in the form \textit{yes}/\textit{no}. If this is the case, for each possible decision a binary variable $x_i\in\lbrace 0,1 \rbrace$ will be used, where if $x_i=1$ it will mean that decision \textit{i} is \textit{yes} and if $x_i=0$  it will mean that decision \textit{i} is \textit{no}.

Once the variables have been identified the next step is to express both the objective of the optimization problem, the objective function, and the constraints that a valid solution must satisfy as mathematical expressions. This mathematical expressions are functions of the binary variables defined above. 

The general formulation of a linear binary program is then:
$$\max_{x}\hspace{2mm} c^T\cdot x \Leftrightarrow \min_{x}\hspace{2mm} -c^T\cdot x$$
\hspace{6cm}subject to:
\begin{equation*}
\begin{split}
&A\cdot x\leq b \\
&Aeq\cdot x = beq \\
&x_i \in\lbrace 0,1 \rbrace\hspace{2mm} \forall \hspace{2mm} i = 1,...,n
\end{split}
\end{equation*}

Where $x$ is a vector containing all the binary variables that represent the \textit{yes}/\textit{no} decisions that can be made in a particular problem, $c^T\cdot x$ is the linear function to be optimized and $A$ and $D$ are the matrices obtained from expressing the inequality and equality constraints, respectively, in a matricial form.

In this particular case the \textit{yes}/\textit{no} decisions that have to be made are if an item is taken and stored in the knapsack or if it is not. Hence the number of variables will be equal to the number of objects considered in the problem, as there is one decision to make for each present object. Once the variables have been defined the constraints can also be expressed as functions of this variables.

Bearing this in mind, the formulation of the stated problem as a binary program is:
 
$$\max_{x_i} \hspace{2mm} f=\sum_{i=1}^{n}v_i\cdot x_i \Leftrightarrow \min_{x_i}\hspace{2mm}f=-\sum_{i=1}^{n}v_i\cdot x_i$$
\hspace{4.7cm}subject to:
\begin{equation*}
\begin{split}
&-x_1-x_2-x_3\leq -1 \\
&-x_4-x_5-x_6\leq -1 \\
&x_{14}+x_{15} \leq 1 \\
&x_1+x_5+x_{10}+x_{20}\leq 3 \\
&\sum_{i=1}^{20}x_i\leq 9  \\
&\sum_{i=1}^{20}w_i\cdot{x_i}\leq W \\
&x_1 + x_4 + x_{16} + x_{17} + x_{18} = 2 \\
&x_i \in\lbrace 0,1 \rbrace\hspace{2mm} \forall \hspace{2mm} i = 1,...,20
\end{split}
\end{equation*}

Where $x_i$ are the decisions of taking ($x_i=1$) or not ($x_i=0$) object $i$, $w_i$ and $v_i$ are respectively the weight and value of object $i$ and $W$ is the maximum weight that can be carried on the knapsack. 
    
\section{Using the MATLAB function \texttt{binprog()} or \texttt{intlinprog()}, write a script code to solve the problem.}

A possible MATLAB implementation that solves the stated binary knapsack problem is:

\begin{lstlisting}[language=Matlab,frame=none]
% Object data
w=[ 2 5 1 4 5 6 5 8 5 4 3 4 5 1 5 4 3 5 2 1]; 				% weights
v=[ 6 5 3 5 3 5 4 4 2 3 4 8 6 5 4 1 2 4 9 5]; 				% values
W = 30;                                       				% Knapsack capacity

% Inequality constraints
A = zeros(6,length(w));                        				% A*x <= b
A(1,1:3) = -1;
A(2,4:6) = -1;
A(3,14:15) = 1;
A(4,1)=1; A(4,5)=1; A(4,10)=1; A(4,20)=1;
A(5,:) = 1;
A(6,:) = w;
b=[-1; -1; 1; 4; 9; W];

% Equality constraints
Aeq = zeros(1,length(w));                      				% Aeq*x = beq
Aeq(1,1)=1; Aeq(1,4)=1; Aeq(1,16:18)=1;
beq=2;

[x,fval] = bintprog(-v',A,b,Aeq,beq);						% Solve program
\end{lstlisting}

Note that, as the MATLAB function \texttt{bintprog()} minimizes the given objective function, in order to maximize the total value the objective function that has to be passed to the \texttt{bintprog()} function is the negative of the total knapsack value computation.  
\newpage
The values of the matrices \texttt{A} and \texttt{b} associated with the inequality constraints are:
\begin{lstlisting}[language=Matlab,frame=none]
A =  -1  -1  -1   0   0   0   0   0   0   0   0   0   0   0   0   0   0   0   0   0   
      0   0   0  -1  -1  -1   0   0   0   0   0   0   0   0   0   0   0   0   0   0
      0   0   0   0   0   0   0   0   0   0   0   0   0   1   1   0   0   0   0   0
      1   0   0   0   1   0   0   0   0   1   0   0   0   0   0   0   0   0   0   1
      1   1   1   1   1   1   1   1   1   1   1   1   1   1   1   1   1   1   1   1
      2   5   1   4   5   6   5   8   5   4   3   4   5   1   5   4   3   5   2   1
     
b' = -1  -1   1   4   9   30
\end{lstlisting}

The values of \texttt{Aeb} and \texttt{beq} associated with the equality constraints are:
\begin{lstlisting}[language=Matlab,frame=none]
Aeq = 1   0   0   1   0   0   0   0   0   0   0   0   0   0   0   1   1   1   0   0
     
beq = 2
\end{lstlisting}
 
The above code yields the solution:
\begin{lstlisting}[language=Matlab,frame=none]
x' =  1   1   0   1   0   1   0   0   0   0   0   1   1   1   0   0   0   0   1   1

fval =  -54.0000
\end{lstlisting}

To interpret the obtained solution it just has to be remembered that each variable $x_i$ of the binary program was a \textit{yes}/\textit{no} decison regarding whether object \textit{i} made it into the knapsack ($x_i=1$) or not ($x_i=0$). This means that objects that should be taken in order to maximize the value of the knapsack satisfying the specified constraints are: \boldmath{$1$},\boldmath{$2$},\boldmath{$4$},\boldmath{$6$},\boldmath{$12$},\boldmath{$13$},\boldmath{$14$},\boldmath{$19$} and \boldmath{$20$}. Finally, the maximum value that can be obtained with this set of objects and constraints is \boldmath{$54$}.\unboldmath 

    
